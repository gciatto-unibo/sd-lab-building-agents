%\documentclass[handout]{beamer}\mode<presentation>{\usetheme{AMSCesenaPurpleAndGold}}
\documentclass[presentation]{beamer}\mode<presentation>{\usetheme{AMSCesenaPurpleAndGold}}
%%%%

\usepackage{sd-lab-building-agents}
\usepackage{my-listings}
\usepackage{forloop}

\title[L4 -- Building Agents]{Building Agents from scratch}
%
\subtitle[SD]
{Distributed Systems / Laboratory\\\scriptsize Sistemi Distribuiti / Laboratorio}
%
\author[Ciatto \and Omicini]
{\emph{Giovanni Ciatto} \and Andrea Omicini\\
	\texttt{giovanni.ciatto@unibo.it} \and \texttt{andrea.omicini@unibo.it}}
%
\institute[DISI, Univ. Bologna]
{Dipartimento di Informatica -- Scienza e Ingegneria (DISI)\\\textsc{Alma Mater Studiorum} -- Universit{\`a} di Bologna a Cesena}
%
\date[A.Y. 2019/2020]{Academic Year 2019/2020}

\setbeamercovered{transparent}

\begin{document}
	
%\\\\\\\\\\\\\\\\\\\\\
\frame{\titlepage}
%\\\\\\\\\\\\\\\\\\\\\

\section{Wetting your appetite}

\begin{frame}
\frametitle{Motivation \& Lecture Goals}

\begin{itemize}
	\item 
\end{itemize}

\end{frame}

\begin{frame}
\frametitle{Lab 5 Repository on GitLab}

	\begin{itemize}
		\item Examples and exercises described in this lecture are provided by means of the following GitLab repository:
		%
		\begin{center}
			\url{https://gitlab.com/pika-lab/courses/ds/aa1920/lab-05}
		\end{center}
		
		\vfill
		
		\item Clone it on your machine using Git
		%
		\begin{itemize}
		    \item[\$] \texttt{git clone \textit{<repo URL>}}
		\end{itemize}
		
		\vfill
		
		\item Even if a minimal environment simply relying on a text editor + Gradle is sufficient for this lab, we kindly suggest to import the cloned repository into some IDE, e.g. IntelliJ Idea or Eclipse
		%
		%
		\begin{itemize}
		    \item in case of problems in importing the project on IntelliJ, try to downgrade the gradle wrapper
		    %
		    \item[\$] \texttt{./gradlew wrapper \alert{--gradle-version \textit{4.8.1}}}
		\end{itemize}
		
		\vfill
		
		\item In order to be able to submit your exercises, please ensure you requested access to the \href{https://gitlab.com/pika-lab/courses/ds/aa1920}{GitLab group of the course}
	\end{itemize}

\end{frame}

%===============================================================================
\section*{}
%===============================================================================

%\\\\\\\\\\\\\\\\\\\\\
\frame{\titlepage}
%\\\\\\\\\\\\\\\\\\\\\

%%===============================================================================
%\section*{\refname}
%%===============================================================================
%
%%\\\\\\\\\\\\\\\\\\\\\
%%%%%
%%\begin{frame}[t,allowframebreaks]\scriptsize
%\begin{frame}[c]\footnotesize
%\frametitle{\refname}
%\bibliographystyle{apalike}
%\bibliography{sd-lab-building-linda}
%\end{frame}
%%\\\\\\\\\\\\\\\\\\\\\

%%%%%%%%%%%%%%%%%%%%%%%%%%%%%%%%%%%%%%%%%%%%%%%%%%%%%%%%%%%%%%%%%%%%%%%%%%%%%%%
\end{document}
%%%%%%%%%%%%%%%%%%%%%%%%%%%%%%%%%%%%%%%%%%%%%%%%%%%%%%%%%%%%%%%%%%%%%%%%%%%%%%%%

